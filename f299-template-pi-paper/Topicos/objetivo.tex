Este sistema de monitoramento tem como meta desenvolver um software que irá auxiliar os produtores na criação de camarões de alta qualidade e na manutenção eficiente do controle dos cativeiros. Dentre os objetivos específicos, destacam-se:

\begin{enumerate}

\item Monitorar remotamente os cativeiros, garantindo um ambiente aquático de alta qualidade para o desenvolvimento dos camarões.

\item Controlar continuamente os níveis de pH, temperatura e amônia da água, emitindo alertas automáticos sempre que os parâmetros ultrapassarem os limites ideais.

\item Registrar e armazenar dados históricos dos cativeiros para possibilitar análises comparativas e avaliação da qualidade ao longo do tempo.

\item Automatizar o fornecimento de ração, controlando horários e quantidades conforme parâmetros definidos pelo usuário.

\item Fornecer relatórios e gráficos gerados a partir das informações monitoradas, permitindo uma análise detalhada e fundamentada das condições do cultivo.

\item Cadastrar usuários, cativeiros e pontos de monitoramento, facilitando a gestão e o acesso centralizado às informações.

\item Desenvolver uma aplicação móvel intuitiva e acessível, proporcionando uma experiência prática e eficiente ao carcinicultor.

\item Contribuir para a sustentabilidade e eficiência da produção aquícola, reduzindo desperdícios e impactos ambientais.
\end{enumerate}