Em 2015, os 193 líderes mundiais comprometeram-se com os Objetivos de Desenvolvimento Sustentável (ODS) \cite{ODS12}, estabelecidos na Agenda 2030, um plano de ação global voltado à erradicação da pobreza, à proteção do meio ambiente e à promoção da paz e prosperidade em todas as regiões do planeta.

Entre esses objetivos, destacam-se o ODS 12 (Produção e Consumo Sustentáveis) e o ODS 14 (Vida na Água). O ODS 12 busca garantir padrões sustentáveis de produção e consumo, assegurando a saúde humana e reduzindo os impactos ambientais. Já o ODS 14 tem como meta a conservação e o uso sustentável dos oceanos, mares e recursos marinhos, combatendo a poluição, a degradação de habitats e a disseminação de doenças.

Dentro desse contexto de sustentabilidade e preservação dos ecossistemas aquáticos, destaca-se a carcinicultura, atividade aquícola voltada à criação de camarões. Segundo Bertini \cite{bertini2021}, Caridea é um grupo de crustáceos amplamente conhecido como camarões, pertencentes à ordem Decapoda, e encontrados tanto em ambientes de água doce quanto salgada. Estima-se a existência de aproximadamente 3.500 espécies, distribuídas em 16 superfamílias e 36 famílias, das quais cerca de 800 vivem em água doce e 2.700 em água salgada. A carcinicultura, por sua vez, refere-se à técnica de criação desses animais em viveiros, podendo ser aplicada a espécies marinhas ou dulcícolas.

De acordo com Rocha \cite{Rocha2023}, a produção mundial de camarão tem crescido a uma taxa média de 4,97% nos últimos sete anos. O Equador é atualmente o maior produtor e exportador mundial, com uma produção estimada de 1.430.000 toneladas e exportações próximas de 1.215.000 toneladas em 2023. Na sequência, destacam-se países como China, Índia, Vietnã, Indonésia, Tailândia e Brasil \cite{Rocha2024}.

No Brasil, o camarão representa a segunda maior fonte de valor na produção aquícola. O estado do Ceará lidera a produção nacional, com aproximadamente 43.778,493 toneladas registradas em 2022. A região Nordeste responde por cerca de 99% da produção brasileira, totalizando 92.438,960 toneladas no mesmo ano \cite{Ximenes2023}.

O Vale do Ribeira, localizado ao sul do estado de São Paulo, destaca-se como uma região propícia à carcinicultura em razão de suas condições ambientais favoráveis — rios, estuários e manguezais — que contribuem para o desenvolvimento sustentável da atividade. O setor é impulsionado pelo turismo e pela comercialização direta com restaurantes, fazendas e consumidores locais, sendo Cananéia, Eldorado e Sete Barras as principais cidades produtoras.

Os carídeos dulcícolas presentes no Vale do Ribeira pertencem principalmente às famílias Palaemonidae e Atyidae. A família Palaemonidae compreende cerca de 980 espécies, distribuídas entre as subfamílias Pontoniinae e Palaemoninae, com destaque para os gêneros Palaemon, Palaemonetes e Macrobrachium. Já a família Atyidae é composta por aproximadamente 469 espécies, distribuídas em 42 gêneros \cite{bertini2021}.

Apesar do grande potencial produtivo, a carcinicultura ainda enfrenta desafios significativos, como a má gestão dos tanques e deficiências de infraestrutura, que podem resultar em poluição da água, acúmulo de resíduos e proliferação de doenças. Esses fatores comprometem a qualidade do produto final e impactam negativamente produtores e consumidores.

Para superar tais desafios e promover práticas mais eficientes e sustentáveis, torna-se essencial o investimento em tecnologias inteligentes voltadas ao monitoramento e controle ambiental. Nesse contexto, uma abordagem promissora é a implementação de um sistema de monitoramento inteligente baseado em Internet das Coisas (IoT).

O sistema proposto visa monitorar de forma eficiente as condições dos tanques de criação de camarões, acompanhando parâmetros críticos como temperatura, pH, níveis de oxigênio dissolvido e amônia. Sensores especializados conectados a microcontroladores, como o Arduino com conectividade Wi-Fi, realizam a coleta e transmissão dos dados em tempo real, garantindo o controle preciso do ambiente aquático.

A solução inclui também um aplicativo móvel, que oferece ao usuário uma interface gráfica intuitiva para visualização dos dados e controle remoto das operações. Além disso, será integrado um alimentador automático programável, permitindo definir horários e quantidades específicas de ração, com sensores de nível que otimizam o fornecimento de alimento.

Como diferencial, o sistema incorpora um mecanismo de recomendação inteligente, capaz de fornecer orientações personalizadas ao carcinicultor com base nos dados coletados. Esse recurso possibilita a geração de relatórios históricos em PDF, gráficos analíticos dos sensores, previsão de variações nos parâmetros e notificações preventivas em situações críticas, como alterações bruscas de temperatura ou níveis inadequados de alimentação. Dessa forma, o sistema atua não apenas no monitoramento, mas também na antecipação de riscos e apoio à tomada de decisão, promovendo maior eficiência e segurança na produção.

Portanto, este trabalho propõe o desenvolvimento de um sistema inteligente de monitoramento e recomendação para a carcinicultura, alinhado aos Objetivos de Desenvolvimento Sustentável, especialmente os ODS 12 e 14, com o intuito de otimizar o manejo produtivo, reduzir impactos ambientais e impulsionar a sustentabilidade da atividade no Vale do Ribeira.