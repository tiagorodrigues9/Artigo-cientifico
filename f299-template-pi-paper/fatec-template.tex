\documentclass[
  a4paper,
  12pt,
  english,
  brazilian,
]{article}

\usepackage{pdflscape}
\usepackage{pdfpages}
\usepackage[]{fatec-article}
\Author{1}{Name={Kusaka.J \\ Muniz.L\\ Ferrari.M\\ Rodrigues.T\\ Roder.V}}

\Author{2}{Name={\{ joao.kusaka@fatec.sp.gov.br \} \\ \{ leandro.muniz@fatec.sp.gov.br \} \\ \{ matheus.abrahao@fatec.sp.gov.br \} \\ \{ tiago.rodrigues39@fatec.sp.gov.br \} \\ \{victor.roder@fatec.sp.gov.br \}}}

\Keyword{1}{Carcinicultura}{Shrimp Farming}
\Keyword{2}{Monitoramento}{Monitoring}
\Keyword{3}{Sustentável}{Sustainability}

\begin{Abstract}[brazilian]
  A carcinicultura é uma atividade de grande relevância econômica e ambiental no Brasil, com destaque para o Nordeste, onde o país figura entre os principais produtores de camarão. Contudo, a criação enfrenta desafios consideráveis, como a má gestão dos tanques, o que pode resultar em poluição, aumento de doenças e, consequentemente, uma queda na qualidade do produto. Para mitigar esses problemas e ao mesmo tempo promover práticas sustentáveis alinhadas aos Objetivos de Desenvolvimento Sustentável (ODS) 12 e 14, propõe-se um sistema de monitoramento contínuo, que combina arduínos e Inteligência Artificial. Esse sistema monitora parâmetros essenciais da água, incluindo temperatura, pH e níveis de amônia, além de automatizar a alimentação dos camarões de forma precisa e eficiente. Os dados coletados são armazenados, permitindo análises históricas e melhorias contínuas no manejo dos cativeiros. A plataforma web oferece aos usuários acesso em tempo real aos dados de cada tanque, além de funcionalidades para gerenciamento de informações e controle automatizado da alimentação, garantindo maior comodidade e precisão na operação. Com esses recursos, o sistema propõe uma solução que visa não apenas melhorar a eficiência da produção, mas também reduzir o impacto ambiental, contribuindo para uma carcinicultura mais responsável e sustentável, e apresentando-se como alternativa de grande potencial para produtores em outras regiões do país e no exterior.

\end{Abstract}

\begin{Abstract}[english]
  Shrimp farming is an economically and environmentally significant activity in Brazil, particularly in the Northeast, where the country stands out as one of the major shrimp producers. However, this industry faces considerable challenges, such as poor tank management, which can lead to water pollution, disease proliferation, and a decline in shrimp quality. To address these issues and promote sustainable practices aligned with Sustainable Development Goals (SDGs) 12 and 14, a continuous 24-hour monitoring system is proposed, combining Arduinos and Artificial Intelligence (AI). This system monitors essential water parameters, including temperature, pH, and ammonia levels, while automating shrimp feeding with precision and efficiency. The collected data is stored, enabling historical analyses and ongoing improvements in tank management. The web platform associated with the system provides users with real-time access to data from each tank, as well as management features and automated feeding control, ensuring greater convenience and operational accuracy. Through these features, the system offers a high-quality solution designed not only to improve production efficiency but also to reduce environmental impact, contributing to a more responsible and sustainable shrimp farming industry. This innovative approach presents significant potential for adoption by producers in other regions of the country and abroad.
\end{Abstract}

\makeindex

\addbibresource{fatec-article.bib}

\begin{document}

\section*{Introdução}%
\label{sect:intro}
No ano de 1945, após a Segunda Guerra Mundial, foi fundada a Organização das Nações Unidas (ONU) com o intuito de garantir a paz e a segurança para o planeta por meio da colaboração dos países. Atualmente, conta com o total de 193 estados que compõem a Assembleia Geral. No ano de 2015, os 193 líderes mundiais se comprometeram com os Objetivos de Desenvolvimento Sustentável (ODS) \cite{ODS12}na Agenda 2030, um plano de desenvolvimento para acabar com a pobreza, cuidar do meio ambiente e do clima, e garantir que as pessoas, em qualquer região do planeta, desfrutem da paz e prosperidade. 

Entre os Objetivos de Desenvolvimento Sustentável, destacam-se os ODS 12 (Produção e Consumo Sustentáveis) e ODS 14 (Oceanos, Mares e Recursos Marinhos). A ODS 12 diz respeito à garantia da qualidade de produção e consumo de alimentos, visando à saúde humana e à diminuição dos impactos ambientais que podem ser causados. A ODS 14, por sua vez, busca a proteção de animais e ecossistemas marinhos, com o intuito de prevenir problemas como poluição, doenças e destruição de habitats naturais.

A carcinicultura, técnica de criação de camarões em viveiros, é amplamente praticada no Brasil. De acordo com \cite{Rocha2023}, a produção de camarão vem crescendo a uma taxa de 4,97\% nos últimos 7 anos. O Equador é atualmente o maior produtor e exportador mundial, com uma produção aproximada de 1.430.000 toneladas e exportação de cerca de 1.215.000 toneladas apenas em 2023, seguido por China, Índia, Vietnã, Indonésia, Tailândia e Brasil \cite{Rocha2024}. No Brasil, o camarão representa a segunda maior fonte de valor entre as espécies cultivadas. O Ceará se destaca como o maior produtor, com 43.778.493 toneladas produzidas em 2022, e o Nordeste responde por cerca de 99\% da produção total de camarão, que atingiu 92.438,960 toneladas no mesmo ano \cite{Ximenes2023}.

O Vale do Ribeira, localizado no sul do estado de São Paulo, tem como uma de suas principais atividades econômicas a criação de camarões. A região, com seu clima favorável e presença de rios, estuários e manguezais, oferece um ambiente adequado para a carcinicultura. A produção local é impulsionada tanto por turistas que adquirem camarões diretamente dos produtores quanto por restaurantes e outras fazendas.

Os carídeos dulcícolas (crustáceos de água doce), representados principalmente pelas famílias Palaemonidae e Atyidae, também são encontrados na região do Vale do Ribeira. A família Palaemonidae inclui cerca de 980 espécies e está dividida em duas subfamílias: Pontoniinae e Palaemoninae. Esta última possui 18 gêneros, dos quais Palaemon, Palaemonetes e Macrobrachius são os mais representativos. A família Atyidae é composta por 469 espécies distribuídas em 42 gêneros \cite{bertini2021}.

Entretanto, a criação de camarões enfrenta desafios significativos, como a falta de cuidado com os tanques e a infraestrutura, o que pode resultar em poluição da água, acúmulo de resíduos e proliferação de doenças entre os camarões. A má gestão dos tanques afeta negativamente a qualidade e a produtividade da criação, prejudicando tanto o consumidor quanto o produtor.

Para enfrentar esses desafios e melhorar a qualidade da produção, torna-se necessário investir em tecnologias avançadas. Uma solução promissora é a implementação de um sistema de monitoramento por meio de **Inteligência Artificial (IA)**. A **IA** é um campo de estudo que visa o desenvolvimento de sistemas capazes de realizar tarefas que exigem inteligência humana. Por meio de algoritmos e cálculos matemáticos, a **IA** pode aprender com base em dados, reconhecendo padrões e tomando decisões de forma autônoma.

Neste contexto, o projeto proposto visa à criação de um sistema de monitoramento automatizado utilizando IA e sistemas embarcados. Esse sistema será responsável por monitorar as condições dos tanques de criação, incluindo parâmetros como temperatura, pH, oxigênio e níveis de amônia. O objetivo não é apenas a coleta de dados, mas também a análise preditiva dessas informações, permitindo o ajuste automático de condições para otimizar a saúde dos camarões e prevenir problemas antes que se tornem críticos. Isso resultaria em uma melhoria significativa na gestão dos viveiros, reduzindo o risco de doenças e aumentando a produtividade de forma sustentável.


\section*{OBJETIVO}\label{sect:obj}

Este sistema de monitoramento tem como meta auxiliar os produtores para criarem camarões de boa qualidade e manter o controle do cativeiro. Dentre os objetivos estão: 	 

\begin{enumerate}

\item Criar um ambiente de monitoramento remoto por 24 horas para total cuidado dos cativeiros. 

\item Manter o controle de pH da água alertando para caso ocorra algum tipo de mudança no pH, assim criando um ambiente de boa qualidade para os camarões impedindo que fiquem com problemas de desenvolvimento. 

\item Monitorar a temperatura da água do cativeiro para manter o ambiente estável para o camarão. 

\item Monitorar os níveis de amônia na água para a prevenção de condições tóxicas para os camarões. 

\item Armazenar dados dos cativeiros e dos camarões para futuras comparações de qualidade. 

\item Controle de horário e automatização da alimentação dos camarões para boa nutrição.

\end{enumerate}

\section*{ESTADO DA ARTE}\label{sect:estadoarte}

A realização de um sistema de monitoramento para cuidado dos camarões é de grande importância, ajuda tanto o produtor a vender um produto de qualidade, tanto o consumidor que irá consumir algo de qualidade sem prejudicar a sua saúde. Por conta desta importância, é de se imaginar que já possam existir sistemas parecidos. Dentre os projetos já existentes, há de se destacar determinados artigos.

O trabalho\cite{dantas2020} é proposto um sistema de telemonitoramento e automação baseado em rede LoRaWAN. O artigo descreve o desenvolvimento de um sistema de monitoramento de temperatura e pH e automação de viveiros de camarão utilizando tecnologia LoRa. A plataforma visa auxiliar carcinicultores, engenheiros de pesca e colaboradores das fazendas de criação no cultivo de camarões. A comunicação entre os nós sensores e o gateway é realizada via protocolo LoRaWAN, permitindo a transferência de dados para a nuvem através de um broker MQTT.

Os resultados registraram as médias de cada setor trabalhado. A temperatura obteve a média de 0 a 80°C. O pH apresentou uma média entre 6,64 e 7,5.

\cite{Uddin2020} diz a respeito de um sistema de monitoramento de fazendas de camarão de água doce baseado em IoT. O sistema tem a proposta de monitorar a qualidade da água como temperatura, pH, salinidade, oxigênio dissolvido e turbidez e verificar o peso, tamanho e porcentagem de sobrevivência por meio de IoT. Isso porque envolve a coleta de dados de sensores em tempo real, armazenamento desses dados na nuvem e disponibilização de visualização e análise desses dados por meio de uma aplicação web. A camada física do sistema utiliza sensores para coletar dados e uma placa arduíno para processamento e transmissão desses dados para a nuvem.

Os resultados indicaram que o sistema funciona de maneira eficiente, com bons níveis de precisão, a temperatura obteve 98,\%, nível de pH obteve 98,24\%, salinidade obteve 94,12\%, oxigênio obteve 95,46\%, turbidez obteve 93,55\%. Os camarões atingiram tamanhos entre 9 e 12 cm, peso varia entre 10 e 20g e a taxa de sobrevivência obteve 90\%. 

\cite{Zainuddin2019} apresenta um sistema de monitoramento da qualidade da água para cultivo de camarão Vannamae baseado em rede de sensores sem fios. O sistema proposto consiste em sensores para monitorar temperatura, pH e turbidez da água, integrados a microcontroladores e dispositivos de comunicação sem fio em conjunto de IoT. Os resultados do estudo incluem aspectos de hardware e software, destacando a produção de unidades transmissoras e receptores, bem como os testes de calibração dos sensores.

Os resultados indicam que o sistema pode funcionar bem, com altos níveis de precisão nos sensores de temperatura com 97,76\%, pH com 98,85\% e turbidez com 99,73\%.

Em geral, os três projetos se assemelham em diversos fatores, como o monitoramento de temperatura e pH da água, a criação de um sistema de monitoramento e o uso de IoT. Um dos três artigos citados trabalha acerca do peso e tamanho do camarão. 

Em comparação com os artigos anteriores, pode-se notar a diferença encontrada em nosso projeto. Ao invés de apenas o uso de IoT, contará com o uso de IA para auxiliar em tarefas como coleta de dados pelo hardware, criação de relatórios a partir das informações coletadas, analisar os dados para alertas sobre possíveis riscos caso ocorra uma mudança tanto nas condições dos tanques tanto na alimentação. Irá controlar a alimentação dos camarões para uma alimentação saudável com comestíveis apropriados para seu consumo, sua alimentação irá se basear em estimativas de crescimento e sobrevivência dos camarões nos cativeiros, caso não sejam controlados adequadamente, geram perdas econômicas e problemas nos cativeiros. Irá monitorar os níveis de amônia e irá realizar comparações com outros cativeiros.

\section*{METODOLOGIA}\label{sect:metodologia}

Neste projeto, foram utilizados uma série de ferramentas e técnicas para auxiliar no desenvolvimento do web. Estas abordagens incluem o uso do Figma, uma plataforma colaborativa para design de interfaces e protótipos pertencente à empresa Figma, INC, do Modelo de Negócios Canvas, uma ferramenta de planejamento estratégico, do DER (Diagrama de Entidade e Relacionamento) e o Oracle Apex, uma plataforma de desenvolvimento de aplicações Low Code.

O Figma será empregado como uma plataforma colaborativa de design para desenvolver os protótipos da interface do usuário do aplicativo. Isso possibilitará que a equipe de design trabalhe de forma conjunta, recebendo feedback instantâneo. Dessa forma, asseguraremos que a interface do usuário seja intuitiva e visualmente atrativa \cite{lopes2023}. 

O Modelo de Negócios Canvas é uma representação detalhada e abrangente da operação, receita e valor que uma empresa proporciona aos seus clientes. Ele visa auxiliar no desenvolvimento da percepção para entendimento da vida em sociedade o papel que lhes é atribuído. No caso específico, o canvas será utilizado para criar uma representação visual do projeto dentro do mercado gastronômico de camarões. Isso ajudará a equipe a identificar parceiros-chave, recursos-chave, atividades-chave e fontes de receita. O objetivo é assegurar que o projeto esteja alinhado tanto com os objetivos científicos quanto com os de negócios \cite{biava2017}. 

A plataforma de criação Apex, será capaz de capacitar os desenvolvedores a criar facilmente aplicativos com funcionalidades, desempenho e experiência. Sem muitas complexidades no desenvolvimento e implantação de aplicativos empresariais. Fornecendo uma interface rica e intuitiva, o resultado é uma aplicação Low Code mais rápida, mais leve e com custos menores. 

Utilizaremos a plataforma para gerenciar informações de registros como dados de usuários, localização de cativeiros, restrições necessárias 

Em comparação com aplicativos de camada intermediária, a execução de aplicativos no Oracle Apex consome muito menos recursos. Esses aplicativos muitas vezes fazem milhares de chamadas para acessar dados no banco de dados, a fim de renderizar uma única tela. Essas chamadas de SQL da camada intermediária para o banco de dados em geral são 10 vezes mais lentas do que quando executadas diretamente dentro do banco de dados. 

O resultado é que os aplicativos do Oracle APEX requerem menos recursos de banco de dados e 100 vezes menos recursos de hardware de camada intermediária em comparação com aplicativo \cite{oracle2024}.

O Diagrama de Entidade e Relacionamento (DER) é uma ferramenta fundamental na modelagem de dados, empregada para representar entidades, seus atributos e os relacionamentos entre elas em um sistema. As entidades são objetos do sistema que possuem atributos, que são características ou propriedades que as descrevem. Os relacionamentos, por sua vez, indicam as conexões ou associações entre essas entidades. No contexto do desenvolvimento do aplicativo, o DER será utilizado para estruturar o banco de dados, definindo entidades como doenças, análises, resultados obtidos, usuários, entre outras, e especificando os relacionamentos entre elas. Essa modelagem é crucial para garantir a eficiência no armazenamento e na recuperação de dados relacionados às doenças e às imagens associadas \cite{awari2023}.

\subparagraph*{\textbf{HTML, CSS, JAVASCRIPT e BOOTSTRAP}}

Será utilizado o HTML (Linguagem de Marcação de Hipertexto) para estruturar e organizar o conteúdo da pagina web. Um hipertexto é um texto usado para fazer referência a outros textos. com o HTML, os usuários podem criar e estruturar seções, parágrafos e links usando elementos, tags e atributos \cite{longen2023}.  
O CSS (Folha de Estilo em Cascata) é uma linguagem de estilo que foi usada para definir a aparência visual de uma página web. Ela é comumente utilizada para atribuir cores, fontes, tamanhos de texto e layouts. O CSS permite que nós separemos a aparência visual da página do conteúdo HTML, permitindo que criemos páginas web mais estéticas \cite{ariane2022}. 

O Javascript é uma linguagem de programação de alto nível que será utilizada para criar interatividade na web. Ele usualmente é incrementado para criar efeitos animados, mapas, gráficos, menus drop-down entre outros. O projeto atual utiliza HTML, CSS e JavaScript para desenvolver o site da equipe e do projeto de forma responsiva e interativa, onde apresentamos a nossa equipe como, nossos objetivos, integrantes e os projetos que estão em desenvolvimento, a partir disso o usuário poderá conhecer melhor o serviço e a equipe \cite{carlos2023}. 

O Bootstrap é um framework web com código-fonte aberto para desenvolvimento de componentes de interface e front-end para sites e aplicações web, usando HTML, CSS e JavaScript e tem como objetivo a construção de sites responsivos. O design responsivo garante que todos os elementos da interface funcionem seguindo o conceito mobile first, ou seja, que o design web inicialmente criado pensando em tablets e smartphones se adapte a outros dispositivos, como desktops \cite{ebac2023}. 

\subparagraph*{\textbf{PHP}}

É um acrônimo recursivo para Hypertext Processor, originalmente Personal Home Page, É uma linguagem de programação interpretada que originalmente foi projetada para criar aplicativos web que funcionam no lado do servidor e geram conteúdo dinâmico na World Wide Web \cite{php2024} . 

Neste projeto o PHP tem a função de guardar dados de usuários no registro de contas para receber atualizações sobre o grupo ou o serviço, e e-mails de respostas ao usuário caso tenha alguma pergunta. 

Além disso, é nessa linguagem que serão elaborados e desenvolvidos os algoritmos para a análise dos dados provenientes dos sensores de monitoramento. Também serão implementados métodos de ordenação para otimizar o uso dos recursos do sistema. 

\subparagraph*{\textbf{SQL}}
  
O MySQL Workbech é um sistema para design de banco de dados que integra design, desenvolvimento, criação e manutenção de SQL em um único ambiente de desenvolvimento. O MySQL Workbech possui um editor SQL que permite que você escreva e execute consultas SQL de forma eficiente, ele oferece vários recursos, facilitando assim a escrita e execução de consultas complexas \cite{andrade2020}. 

\subparagraph*{\textbf{XAMMP}}

O XAMPP (Apache, MySQL, PHP, Pearl) é um software de desenvolvimento web amplamente empregada na indústria, simplificando a instalação e operação de um servidor web \cite{methaseo2023}.

\subparagraph*{\textbf{Inteligência Artificial}}

A Intêligência Artificial (IA) é o campo que estuda o desenvolvimento de sistemas para realizar tarefas. As IAs que serão utilizadas serão: IA de Recomendação; IA de Redes Neurais; IA de Aprendizado de Máquina.

A IA de Recomendação é um software que fornece recomendações com base no perfil do usuário. A partir de um banco de dados, a IA é capaz de garantir que a informação chegue ao usuário de acordo com os fatores estabelecidos.\cite{mensagem2023}

IA de Redes Neurais se baseia em neurônios artificiais conectados entre si que formam redes neurais que são capazes de tomar decisões por contra própria com base em dados.\cite{Aceleradora2023}

IA de Aprendizado de Máquina tem como funcionalidade a capacidade de aprender e aprimorar a sua perfomance com base em dados, assim se tornando maais eficiente e prática.\cite{Aceleradora2023}

\subparagraph*{\textbf{Sistemas Embarcados}}
 
Sistemas embarcados ou embutidos são sistemas computacionais especializados que atuam em conjunto com hardware e software, se responsabilizando por alguma função ou ação específica, no caso desse projeto, o sistema embarcado é representado pelo Arduíno \cite{souza2022}. 

O Arduino é uma plataforma de construção e prototipagem eletrônica que simplifica a criação de projetos interativos para o dia a dia, incluindo aplicações na Internet das Coisas(IoT). O Arduino é um hardware de código aberto, permitindo que qualquer pessoa possa aproveitar de suas funcionalidades. Com uma ampla gama de sensores disponíveis, o Arduino facilita a integração de diferentes componentes ao sistema \cite{thomsen2023}.

Neste projeto, utilizamos sensores de PH e temperatura para medir e controlar a saúde do cativeiro de camarões, a fim de assegurar sua qualidade.  

 

 

\newpage

\printbibliography

\end{document}