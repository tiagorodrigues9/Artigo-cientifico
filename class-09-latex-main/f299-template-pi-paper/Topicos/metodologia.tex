Neste projeto, foram utilizados uma série de ferramentas e técnicas para auxiliar no desenvolvimento do web. Estas abordagens incluem o uso do Figma, uma plataforma colaborativa para design de interfaces e protótipos pertencente à empresa Figma, INC, do Modelo de Negócios Canvas, uma ferramenta de planejamento estratégico, do DER (Diagrama de Entidade e Relacionamento) e o Oracle Apex, uma plataforma de desenvolvimento de aplicações Low Code.

O Figma será empregado como uma plataforma colaborativa de design para desenvolver os protótipos da interface do usuário do aplicativo. Isso possibilitará que a equipe de design trabalhe de forma conjunta, recebendo feedback instantâneo. Dessa forma, asseguraremos que a interface do usuário seja intuitiva e visualmente atrativa \cite{lopes2023}. 

O Modelo de Negócios Canvas é uma representação detalhada e abrangente da operação, receita e valor que uma empresa proporciona aos seus clientes. Ele visa auxiliar no desenvolvimento da percepção para entendimento da vida em sociedade o papel que lhes é atribuído. No caso específico, o canvas será utilizado para criar uma representação visual do projeto dentro do mercado gastronômico de camarões. Isso ajudará a equipe a identificar parceiros-chave, recursos-chave, atividades-chave e fontes de receita. O objetivo é assegurar que o projeto esteja alinhado tanto com os objetivos científicos quanto com os de negócios \cite{biava2017}. 

A plataforma de criação Apex, será capaz de capacitar os desenvolvedores a criar facilmente aplicativos com funcionalidades, desempenho e experiência. Sem muitas complexidades no desenvolvimento e implantação de aplicativos empresariais. Fornecendo uma interface rica e intuitiva, o resultado é uma aplicação Low Code mais rápida, mais leve e com custos menores. 

Utilizaremos a plataforma para gerenciar informações de registros como dados de usuários, localização de cativeiros, restrições necessárias 

Em comparação com aplicativos de camada intermediária, a execução de aplicativos no Oracle Apex consome muito menos recursos. Esses aplicativos muitas vezes fazem milhares de chamadas para acessar dados no banco de dados, a fim de renderizar uma única tela. Essas chamadas de SQL da camada intermediária para o banco de dados em geral são 10 vezes mais lentas do que quando executadas diretamente dentro do banco de dados. 

O resultado é que os aplicativos do Oracle APEX requerem menos recursos de banco de dados e 100 vezes menos recursos de hardware de camada intermediária em comparação com aplicativo \cite{oracle2024}.

O Diagrama de Entidade e Relacionamento (DER) é uma ferramenta fundamental na modelagem de dados, empregada para representar entidades, seus atributos e os relacionamentos entre elas em um sistema. As entidades são objetos do sistema que possuem atributos, que são características ou propriedades que as descrevem. Os relacionamentos, por sua vez, indicam as conexões ou associações entre essas entidades. No contexto do desenvolvimento do aplicativo, o DER será utilizado para estruturar o banco de dados, definindo entidades como doenças, análises, resultados obtidos, usuários, entre outras, e especificando os relacionamentos entre elas. Essa modelagem é crucial para garantir a eficiência no armazenamento e na recuperação de dados relacionados às doenças e às imagens associadas \cite{awari2023}.

\subparagraph*{\textbf{HTML, CSS, JAVASCRIPT e BOOTSTRAP}}

Será utilizado o HTML (Linguagem de Marcação de Hipertexto) para estruturar e organizar o conteúdo da pagina web. Um hipertexto é um texto usado para fazer referência a outros textos. com o HTML, os usuários podem criar e estruturar seções, parágrafos e links usando elementos, tags e atributos \cite{longen2023}.  
O CSS (Folha de Estilo em Cascata) é uma linguagem de estilo que foi usada para definir a aparência visual de uma página web. Ela é comumente utilizada para atribuir cores, fontes, tamanhos de texto e layouts. O CSS permite que nós separemos a aparência visual da página do conteúdo HTML, permitindo que criemos páginas web mais estéticas \cite{ariane2022}. 

O Javascript é uma linguagem de programação de alto nível que será utilizada para criar interatividade na web. Ele usualmente é incrementado para criar efeitos animados, mapas, gráficos, menus drop-down entre outros. O projeto atual utiliza HTML, CSS e JavaScript para desenvolver o site da equipe e do projeto de forma responsiva e interativa, onde apresentamos a nossa equipe como, nossos objetivos, integrantes e os projetos que estão em desenvolvimento, a partir disso o usuário poderá conhecer melhor o serviço e a equipe \cite{carlos2023}. 

O Bootstrap é um framework web com código-fonte aberto para desenvolvimento de componentes de interface e front-end para sites e aplicações web, usando HTML, CSS e JavaScript e tem como objetivo a construção de sites responsivos. O design responsivo garante que todos os elementos da interface funcionem seguindo o conceito mobile first, ou seja, que o design web inicialmente criado pensando em tablets e smartphones se adapte a outros dispositivos, como desktops \cite{ebac2023}. 

\subparagraph*{\textbf{PHP}}

É um acrônimo recursivo para Hypertext Processor, originalmente Personal Home Page, É uma linguagem de programação interpretada que originalmente foi projetada para criar aplicativos web que funcionam no lado do servidor e geram conteúdo dinâmico na World Wide Web \cite{php2024} . 

Neste projeto o PHP tem a função de guardar dados de usuários no registro de contas para receber atualizações sobre o grupo ou o serviço, e e-mails de respostas ao usuário caso tenha alguma pergunta. 

Além disso, é nessa linguagem que serão elaborados e desenvolvidos os algoritmos para a análise dos dados provenientes dos sensores de monitoramento. Também serão implementados métodos de ordenação para otimizar o uso dos recursos do sistema. 

\subparagraph*{\textbf{SQL}}
  
O MySQL Workbech é um sistema para design de banco de dados que integra design, desenvolvimento, criação e manutenção de SQL em um único ambiente de desenvolvimento. O MySQL Workbech possui um editor SQL que permite que você escreva e execute consultas SQL de forma eficiente, ele oferece vários recursos, facilitando assim a escrita e execução de consultas complexas \cite{andrade2020}. 

\subparagraph*{\textbf{XAMMP}}

O XAMPP (Apache, MySQL, PHP, Pearl) é um software de desenvolvimento web amplamente empregada na indústria, simplificando a instalação e operação de um servidor web \cite{methaseo2023}.

\subparagraph*{\textbf{Inteligência Artificial}}

A Intêligência Artificial (IA) é o campo que estuda o desenvolvimento de sistemas para realizar tarefas. As IAs que serão utilizadas serão: IA de Recomendação; IA de Redes Neurais; IA de Aprendizado de Máquina.

A IA de Recomendação é um software que fornece recomendações com base no perfil do usuário. A partir de um banco de dados, a IA é capaz de garantir que a informação chegue ao usuário de acordo com os fatores estabelecidos.\cite{mensagem2023}

IA de Redes Neurais se baseia em neurônios artificiais conectados entre si que formam redes neurais que são capazes de tomar decisões por contra própria com base em dados.\cite{Aceleradora2023}

IA de Aprendizado de Máquina tem como funcionalidade a capacidade de aprender e aprimorar a sua perfomance com base em dados, assim se tornando maais eficiente e prática.\cite{Aceleradora2023}

\subparagraph*{\textbf{Sistemas Embarcados}}
 
Sistemas embarcados ou embutidos são sistemas computacionais especializados que atuam em conjunto com hardware e software, se responsabilizando por alguma função ou ação específica, no caso desse projeto, o sistema embarcado é representado pelo Arduíno \cite{souza2022}. 

O Arduino é uma plataforma de construção e prototipagem eletrônica que simplifica a criação de projetos interativos para o dia a dia, incluindo aplicações na Internet das Coisas(IoT). O Arduino é um hardware de código aberto, permitindo que qualquer pessoa possa aproveitar de suas funcionalidades. Com uma ampla gama de sensores disponíveis, o Arduino facilita a integração de diferentes componentes ao sistema \cite{thomsen2023}.

Neste projeto, utilizamos sensores de PH e temperatura para medir e controlar a saúde do cativeiro de camarões, a fim de assegurar sua qualidade.  

 

 