No ano de 1945, após a Segunda Guerra Mundial, foi fundada a Organização das Nações Unidas (ONU) com o intuito de garantir a paz e a segurança para o planeta por meio da colaboração dos países. Atualmente, conta com o total de 193 estados que compõem a Assembleia Geral. No ano de 2015, os 193 líderes mundiais se comprometeram com os Objetivos de Desenvolvimento Sustentável (ODS) \cite{ODS12}na Agenda 2030, um plano de desenvolvimento para acabar com a pobreza, cuidar do meio ambiente e do clima, e garantir que as pessoas, em qualquer região do planeta, desfrutem da paz e prosperidade. 

Entre os Objetivos de Desenvolvimento Sustentável, destacam-se os ODS 12 (Produção e Consumo Sustentáveis) e ODS 14 (Oceanos, Mares e Recursos Marinhos). A ODS 12 diz respeito à garantia da qualidade de produção e consumo de alimentos, visando à saúde humana e à diminuição dos impactos ambientais que podem ser causados. A ODS 14, por sua vez, busca a proteção de animais e ecossistemas marinhos, com o intuito de prevenir problemas como poluição, doenças e destruição de habitats naturais.

A carcinicultura, técnica de criação de camarões em viveiros, é amplamente praticada no Brasil. De acordo com \cite{Rocha2023}, a produção de camarão vem crescendo a uma taxa de 4,97\% nos últimos 7 anos. O Equador é atualmente o maior produtor e exportador mundial, com uma produção aproximada de 1.430.000 toneladas e exportação de cerca de 1.215.000 toneladas apenas em 2023, seguido por China, Índia, Vietnã, Indonésia, Tailândia e Brasil \cite{Rocha2024}. No Brasil, o camarão representa a segunda maior fonte de valor entre as espécies cultivadas. O Ceará se destaca como o maior produtor, com 43.778.493 toneladas produzidas em 2022, e o Nordeste responde por cerca de 99\% da produção total de camarão, que atingiu 92.438,960 toneladas no mesmo ano \cite{Ximenes2023}.

O Vale do Ribeira, localizado no sul do estado de São Paulo, tem como uma de suas principais atividades econômicas a criação de camarões. A região, com seu clima favorável e presença de rios, estuários e manguezais, oferece um ambiente adequado para a carcinicultura. A produção local é impulsionada tanto por turistas que adquirem camarões diretamente dos produtores quanto por restaurantes e outras fazendas.

Os carídeos dulcícolas (crustáceos de água doce), representados principalmente pelas famílias Palaemonidae e Atyidae, também são encontrados na região do Vale do Ribeira. A família Palaemonidae inclui cerca de 980 espécies e está dividida em duas subfamílias: Pontoniinae e Palaemoninae. Esta última possui 18 gêneros, dos quais Palaemon, Palaemonetes e Macrobrachius são os mais representativos. A família Atyidae é composta por 469 espécies distribuídas em 42 gêneros \cite{bertini2021}.

Entretanto, a criação de camarões enfrenta desafios significativos, como a falta de cuidado com os tanques e a infraestrutura, o que pode resultar em poluição da água, acúmulo de resíduos e proliferação de doenças entre os camarões. A má gestão dos tanques afeta negativamente a qualidade e a produtividade da criação, prejudicando tanto o consumidor quanto o produtor.

Para enfrentar esses desafios e melhorar a qualidade da produção, torna-se necessário investir em tecnologias avançadas. Uma solução promissora é a implementação de um sistema de monitoramento por meio de **Inteligência Artificial (IA)**. A **IA** é um campo de estudo que visa o desenvolvimento de sistemas capazes de realizar tarefas que exigem inteligência humana. Por meio de algoritmos e cálculos matemáticos, a **IA** pode aprender com base em dados, reconhecendo padrões e tomando decisões de forma autônoma.

Neste contexto, o projeto proposto visa à criação de um sistema de monitoramento automatizado utilizando IA e sistemas embarcados. Esse sistema será responsável por monitorar as condições dos tanques de criação, incluindo parâmetros como temperatura, pH, oxigênio e níveis de amônia. O objetivo não é apenas a coleta de dados, mas também a análise preditiva dessas informações, permitindo o ajuste automático de condições para otimizar a saúde dos camarões e prevenir problemas antes que se tornem críticos. Isso resultaria em uma melhoria significativa na gestão dos viveiros, reduzindo o risco de doenças e aumentando a produtividade de forma sustentável.
