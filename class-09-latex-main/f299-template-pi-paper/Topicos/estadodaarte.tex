A realização de um sistema de monitoramento para cuidado dos camarões é de grande importância, ajuda tanto o produtor a vender um produto de qualidade, tanto o consumidor que irá consumir algo de qualidade sem prejudicar a sua saúde. Por conta desta importância, é de se imaginar que já possam existir sistemas parecidos. Dentre os projetos já existentes, há de se destacar determinados artigos.

O trabalho \cite{dantas2020} é proposto um sistema de telemonitoramento e automação baseado em rede LoRaWAN. O artigo descreve o desenvolvimento de um sistema de monitoramento de temperatura e pH e automação de viveiros de camarão utilizando tecnologia LoRa. A plataforma visa auxiliar carcinicultores, engenheiros de pesca e colaboradores das fazendas de criação no cultivo de camarões. A comunicação entre os nós sensores e o gateway é realizada via protocolo LoRaWAN, permitindo a transferência de dados para a nuvem através de um broker MQTT.  

Os resultados registraram as médias de cada setor trabalhado. A temperatura obteve a média de 0 a 80°C. O pH apresentou uma média entre 6,64 e 7,5.

\cite{Uddin2020} diz a respeito de um sistema de monitoramento de fazendas de camarão de água doce baseado em IoT. O sistema tem a proposta de monitorar a qualidade da água como temperatura, pH, salinidade, oxigênio dissolvido e turbidez e verificar o peso, tamanho e porcentagem de sobrevivência por meio de IoT. Isso porque envolve a coleta de dados de sensores em tempo real, armazenamento desses dados na nuvem e disponibilização de visualização e análise desses dados por meio de uma aplicação web. A camada física do sistema utiliza sensores para coletar dados e uma placa arduíno para processamento e transmissão desses dados para a nuvem.

Os resultados indicaram que o sistema funciona de maneira eficiente, com bons níveis de precisão, a temperatura obteve 98,\%, nível de pH obteve 98,24\%, salinidade obteve 94,12\%, oxigênio obteve 95,46\%, turbidez obteve 93,55\%. Os camarões atingiram tamanhos entre 9 e 12 cm, peso varia entre 10 e 20g e a taxa de sobrevivência obteve 90\%. 

\cite{Zainuddin2019} apresenta um sistema de monitoramento da qualidade da água para cultivo de camarão Vannamae baseado em rede de sensores sem fios. O sistema proposto consiste em sensores para monitorar temperatura, pH e turbidez da água, integrados a microcontroladores e dispositivos de comunicação sem fio em conjunto de IoT. Os resultados do estudo incluem aspectos de hardware e software, destacando a produção de unidades transmissoras e receptores, bem como os testes de calibração dos sensores.

Os resultados indicam que o sistema pode funcionar bem, com altos níveis de precisão nos sensores de temperatura com 97,76\%, pH com 98,85\% e turbidez com 99,73\%.

Em geral, os três projetos se assemelham em diversos fatores, como o monitoramento de temperatura e pH da água, a criação de um sistema de monitoramento e o uso de IoT. Um dos três artigos citados trabalha acerca do peso e tamanho do camarão. 

Em comparação com os artigos anteriores, pode-se notar a diferença encontrada em nosso projeto. Ao invés de apenas o uso de IoT, contará com o uso de IA para auxiliar nas tarefas, irá controlar a alimentação dos camarões para uma alimentação saudável tornando-os fortes, irá monitorar os níveis de amônia e irá realizar comparações com outros camarões. 