Este atual projeto tem a meta de cumprir as ODS 12 e ODS 14, que trabalham para a segurança da alimentação humana e proteção de animais marinhos e com os resultados obtidos contribuem para a realização destas metas. Com o monitoramento da água e dos camarões, é realizada a contribuição para boa qualidade e cuidado com o animal e com o meio ambiente, assim ajudando na carcinicultura e seu desenvolvimento.

Com os benefícios deste sistema, o usuário poderá desfrutar de uma plataforma confiável com bom gerenciamento de cada cativeiro de uma maneira simples e de fácil uso, dessa forma ajudando o produtor a controlar a temperatura, pH e amônia da água para o camarão estar em boas condições, além de ajudar na boa alimentação e fornecer dados a respeito de cada tanque e camarões para futuras comparações e análises.

Atualmente, o projeto está em fase de desenvolvimento e busca por melhorias contínuas, com o objetivo de criar um sistema altamente confiável. Futuras melhorias incluem uma investigação mais aprofundada na coleta de dados dos sensores e o uso de inteligência artificial (IA). Também está prevista a otimização da experiência do usuário (UX) e da interface do usuário (UI) do sistema web, com possíveis adições para tornar o sistema mais eficiente e completo.

Sobre termos de viabilidade econômica, é de se esperar que acabe sendo um sistema de alto valor por conta de seu desenvolvimento e instalação no local tendo em base suas funcionalidades e os equipamentos que deverão ser de boa qualidade para um bom uso e durabilidade do produto.

Em relação ao tema, temos o objetivo de dar continuidade na pesquisa nesta área de maneira mais aprofundada para um conhecimento maior e procura de novas ações para o sistema.

