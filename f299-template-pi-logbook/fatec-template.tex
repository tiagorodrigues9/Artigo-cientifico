\documentclass[
landscape,
  a4paper,
  12pt,
  english,
  brazilian,
]{article}

\usepackage[]{fatec-article}
\usepackage{setspace}

\begin{document}

\section*{Instruções para o preenchimento}
\doublespacing
\begin{enumerate}
    \item O Diário de Bordo é usado para registrar atividades, progressos, ideias e desafios enfrentados em um projeto ou durante a rotina de trabalho. Serve como um registro cronológico e detalhado das operações diárias, facilitando a organização e o acompanhamento das tarefas.
    \doublespacing
    \item Durante o registro das atividades deve-se incluir detalhes como datas, horários, descrições de tarefas, nomes de participantes e observações relevantes.  Esta documentação contínua ajuda na avaliação do progresso de projetos ou atividades, permitindo ajustes e melhorias contínuas nos processos.
    \doublespacing
    \item Para evidenciar a realização das tarefas, você poderá utilizar a criação de anexos para adicionar anotações, fotos, prints, questionários, entre outros.
\end{enumerate}

 \begin{table}[]
\centering
 

\begin{tabular}{|>{\raggedright\arraybackslash}p{0.2\linewidth}|l|l|>{\raggedright\arraybackslash}p{0.2\linewidth}|>{\raggedright\arraybackslash}p{0.2\linewidth}|}
\hline
Nome da Atividade & Data de início & Data de término & Responsável pela atividade & Descrição da atividade realizada \\\hline
                  Início dos trabalhos para o projeto&                            06/08/2025&                13/08/2025&                  João Kusaka; Matheus Abrahão; Tiago Rodrigues; Victor Roder; Leandro Augusto;&                                  Início dos planejamentos e realizações para o projeto no 4° Semestre\\ \hline
                  Modificações na aplicação&                            06/08/2025&                29/10/2025&                 João Kusaka; &                                 Processo de alterações na aplicações, dentre as modificações estão: Alteração do código para utilizar React e hospedagem do banco de dados na nuvem no Mongo Atlas. \\ \hline 
 Reunião do P.I&                            11/08/2025&                11/08/2025&                 João Kusaka; Matheus Abrahão; Tiago Rodrigues; Victor Roder; Leandro Augusto;&                                 Reunião realizada para organizar as ideias do projeto, realizar mudanças em relação ao funcionamento do projeto e definir tarefas para cada integrante.\\ \hline
                  Modificações no artigo científico &                            11/08/2025&                29/10/2025&                 Tiago Rodrigues;&                                  Alterações realizadas no artigo para melhorias de escrita e inclusão de mais informações.\\ \hline
                 
\end{tabular}


\end{table}

 \begin{table}[]
\centering


\begin{tabular}{|>{\raggedright\arraybackslash}p{0.2\linewidth}|l|l|>{\raggedright\arraybackslash}p{0.2\linewidth}|>{\raggedright\arraybackslash}p{0.2\linewidth}|}

\hline
 Criação do questionário para a pesquisa de campo 1° versão&                            13/08/2025&                28/08/2025&                 João Kusaka; Matheus Abrahão; Tiago Rodrigues; Victor Roder; Leandro Augusto;&                                 Formulação da primeira versão do questionário para a pesquisa de campo com auxílio da professora Simone.\\ \hline
                  Modificações no figma&                            14/08/2025&               29/10/2025&                 Matheus Ferrari.&                                  Modificações feitas em todas as telas com base nas alterações feitas na aplicação web.\\ \hline
                  Inscrição para o XXXVII Congresso da UNESP de Iniciação Científica& 29/08/2025& 03/09/2025& João Kusaka; Matheus Abrahão; Tiago Rodrigues; Victor Roder; Leandro Augusto &Fomos selecionados para se inscrever no Congresso da UNESP, o depósito dos requisitos foi realizado por Tiago Rodrigues.\\\hline
                  Criação do questionário para pesquisa de campo 2° versão&                            04/09/2025&                05/09/2025&                 João Kusaka; Matheus Abrahão; Tiago Rodrigues; Victor Roder; Leandro Augusto;&                                  Com base nas correções sugeridas feitas pelo professor Frederico e professora Simone, foi feita a revisão e alterações nas questões criando assim a segunda versão do questionário.\\ \hline
                  Modificação no arduíno&                            29/08/2025&                29/10/2025&                  Leandro Augusto;&                                 Alterações no código do arduíno e no modelo 3D do projeto para melhorar a performance e ser de fácil uso.\\ \hline
 
\end{tabular}


\end{table}

 \begin{table}[]
\centering


\begin{tabular}{|>{\raggedright\arraybackslash}p{0.2\linewidth}|l|l|>{\raggedright\arraybackslash}p{0.2\linewidth}|>{\raggedright\arraybackslash}p{0.2\linewidth}|}

\hline

Visita na UNESP Registro para pesquisa de campo& 15/09/2025& 15/09/2025&  Tiago Rodrigues; Victor Roder; Leandro Augusto. &Realizada a primeira pesquisa de campo na UNESP Registro. Foi conduzida pela professora Giovana Bertini e pela aluna Estephany Konesuk\\\hline
 Reunião do P.I& 17/09/2025& 17/09/2025& João Kusaka; Matheus Abrahão; Tiago Rodrigues; Victor Roder; Leandro Augusto;& Realizada uma reunião na FATEC para adição de uma nova funcionalidade no sistema, para organizar pontos importantes e sobre o andamento do projeto.\\\hline
 Inclusão da funcionalidade de tutorial& 18/09/2025& 20/09/2025& João Kusaka.&Realizada a inclusão da funcionalidade de tutorial para facilitar a utilização para o usuário. A funcionalidade foi decidida na aula de Experiência de Usuário.\\ \hline 
 Coleta de camarões& 22/09/2025& 22/09/2025& Leandro Muniz. &Feita a coleta dos camarões que foram doados para o projeto pela professora Giovana Bertini da UNESP Registro\\\hline
                  Entrega dos requisitos para o GrandPrix&                            12/11/2024&                12/11/2024&                 João Kusaka; Matheus Ferrari; Tiago Rodrigues; Victor Roder; Isabele Queiroz.&                                  Realizada a entrega dos requisitos solicitado para o GrandPrix do SENAC, sendo eles um pitch e um figma.\\ \hline 

\end{tabular}


\end{table}

 \begin{table}[]
\centering


\begin{tabular}{|>{\raggedright\arraybackslash}p{0.2\linewidth}|l|l|>{\raggedright\arraybackslash}p{0.2\linewidth}|>{\raggedright\arraybackslash}p{0.2\linewidth}|}

\hline

 Entrega do Estado da Arte, Metodologia e Resultados& 03/11/2024& 03/11/2024& Tiago Rodrigues.&Entrega do Estado da Arte, Metodologia e Resultados a pedido do professor orientador para melhorias e ajustes.\\\hline
                  Modificações no Banner&                            13/11/2024&                17/11/2024&                 Tiago Rodrigues.&                                  Foram realizadas mudanças no banner para estarem com as informações atualizadas acerca do projeto e dos novos protótipos de tela. Banner atualizado pelo Figma.\\ \hline
                  Modificações no Pitch&                            13/11/2024&                17/11/2024&                 Isabele Queiroz.&                                  Foram realizadas mudanças no pitch com objetivo de deixar o pitch atualizado.\\ \hline
                  Realização do modelo físico&                            13/11/2024&                17/11/2024&                 Tiago Rodrigues.&                                  O modelo físico do banco de dados foi realizado por meio da plataforma dbdiagram.io, foi moldado com base no modelo conceitual.\\ \hline                  
\end{tabular}


\end{table}

\end{document}